% arara: pdflatex
% arara: bibtex
% arara: pdflatex
% arara: pdflatex
\documentclass[11pt,a4paper,twoside,openright,draft]{memoir}

\usepackage[utf8]{inputenc}
\usepackage[T1]{fontenc}
\usepackage{verbatim,amsmath,amsfonts,amssymb,amsthm}
\usepackage[english,polish]{babel}


\usepackage{filemod}

\usepackage[a4paper,hyperindex=true]{hyperref}
\hypersetup{%
pdftitle="Homeostaza wapniowa"
}

\usepackage{memhfixc}

% Styl punktowania
\newenvironment{bulletList}%
{ \begin{list}%
	{$\bullet$}%
	{\setlength{\labelwidth}{25pt}%
		\setlength{\leftmargin}{50pt}%
		\setlength{\itemsep}{\parsep}}}%
{ \end{list}}


\usepackage{geometry}
\geometry{
	a4paper,
	total={210mm,297mm},
	left=20mm,
	right=20mm,
	top=20mm,
	bottom=20mm,
}

% Definicje 

\renewcommand*\bookname{Tom}

\settocdepth{section}

% dzielenie wierszy
\widowpenalty=10000
\clubpenalty=10000 
\brokenpenalty=10000

\begin{document}

% ===================================================================
%       FRONT
% ===================================================================

	\frontmatter
	\thispagestyle{empty}

%\title{Inżynieria Biomedyczna. Podstawy i Zastosowania}
%\author{K. Cieślicki, T. Lipniacki, J. Waniewski}
%\date{}
\begin{center}
	
	{\Huge \textbf{\textsc{Inżynieria Biomedyczna}}}
	
	\vspace{1cm}
	
	{\huge  \textbf{Podstawy i~Zastosowania}}
	
	\vspace{3cm}
{\Large Redaktorzy tomu: K. Cieślicki, T. Lipniacki, J. Waniewski}
\end{center}

	\begin{KeepFromToc}
		
		\tableofcontents
%		\clearpage
%		\listoffigures
%		\clearpage
%		\listoftables
	\end{KeepFromToc}

% ===================================================================
%       MAIN
% ===================================================================

	\mainmatter
	\pagestyle{ruled}
%	\OnehalfSpacing
%	\linenumbers
	\book{Modelowanie procesów fizjologicznych i~patologicznych}
		
		\part{Modelowanie matematyczne układów fizjologicznych}
			
			% lepiej, zeby kazdy z artykulow byl w osobnym pliku. Tytul jako \chapter i mozna zachowac sekscje z klasy article
			\chapter{Modelowanie rozkładu i transportu wody i substancji w organizmie pacjenta (M. Dębowska, L. Pstraś, J. Poleszczuk, M. Pietribiasi, J. Piętka-Stachowska, A. Jung)}

Modelowanie kompartmentowe. Farmakokinetyka. Pozaustrojowe oczyszczanie krwi. Kinetyczny model mocznika. Usuwanie mało- i średnio-cząsteczkowych substancji w hemodializie i dializie otrzewnowej. Modele pseudo-jednokompartmentowe: kinetyka fosforanów w hemodializie. Usuwanie makrocząsteczek: dializa otrzewnowa, zabiegi sztucznej wątroby. Usuwanie nadmiaru wody w czasie hemodializy i dializy otrzewnowej. Model regionalnego przepływu krwi. 
			\chapter[Modelowanie fali pulsu w naczyniach tętniczych]{Modelowanie fali pulsu w naczyniach tętniczych\\{\large (J. Poleszczuk)}}
			\chapter{Modelowanie układu krążenia, wybrane aspekty \\(K.~Zieliński)}
			\chapter{Modelowanie krążenia mózgowego (K.~Cieślicki)}
			\chapter{Modelowanie układu oddechowego (T.~Gólczewski)}
			\chapter{Modelowanie interakcji oddechowo-krążeniowej (T.~Gólczewski,~K.~Zieliński)}
			\chapter{Modelowanie metabolizmu mineralnego (M.~Dębowska)}
			\chapter{Modelowanie metabolizmu węglowodanów, układ glukoza – insulina (J.~Waniewski, J.~Śmieja)}
			\chapter{Modelowanie homeostazy cholesterolu (K.~Kubica,~M.~Żulpo,~J.~Balbus)}
			\chapter{Modelowanie motoryki pęcherzyka żółciowego (M.~Żulpo,~J.~Balbus,~K.~Kubica)}
			\chapter{Numeryczne modelowanie przepływu krwi, płynu mózgowo-rdzeniowego i~powietrza w rzeczywistych geometriach (A.~Piechna,~K.~Cieślicki)}
			\chapter{Modelowanie mechaniki struktur tętniczych (K.~Cieślicki,~A.~Piechna)}
			\chapter{Modelowanie procesów słuchowych w celu oceny ryzyka uszkodzeń słuchu (J.~Kotus,~A.~Czyżewski)}
			
			
		\part{Modelowanie matematyczne procesów tkankowych i komórkowych}
		
			\chapter{Modelowanie procesów transportu wody i~substancji na poziomie tkankowym (J.~Stachowska-Piętka)}
			\chapter{Modelowanie transportu światła w tkance (A.~Liebert)}
			\chapter{Modelowanie transportu ciepła w tkance (A.~Nowakowski)}
			\chapter{Modelowanie przepływu krwi przez chirurgiczne zespolenia naczyniowe (Z.~Małota,~J.~Waniewski)}
			\chapter{Modelowanie przepływu krwi przez zastawki serca \\(Z.~Małota)}
			\chapter{Modelowanie motoneuronu (M.~Piotrkiewicz)}
			\chapter{Modelowanie sieci neuronów (T.~Bem)}
			\chapter{Modelowanie odpowiedzi immunologicznej na rozwój nowotworu (J.~Poleszczuk, U.~Foryś)}
	
			\chapter[Modelowanie wapnia komórkowego]{Modelowanie wapnia komórkowego\\{\large (B.~Kaźmierczak)}}

\section{Homeostaza wapniowa w komórce}

Jony wapniowe kontrolują wiele różnorodnych procesów komórkowych, takich jak skurcz mięśni, egzocytozę, transkrypcję a nawet apoptozę. Aby uzyskać tak różnorodny wachlarz możliwości komórki wykorzystują zestawy ,,narzędzi'' białkowych, które skłądają sie na tzw. sygnałosom wapniowy \cite{Berridge2012}. Każdy typ komórek charakteryzuje się specyficznym układem białek sensorycznych i efektorowych, które przekazują informacje w dół kaskady informacyjnej, jaką jest wapniowy szlak sygnałowy.

%Calcium ions regulate processes as diverse as cell motility, gene transcription, muscle contraction, and exocytosis (Berridge et al. 2000). The first realization that they are critical for cellular function is often attributed to Sydney Ringer, who discovered in 1883 that saline solution made up using London tap water (which contained calcium) supported the contraction of isolated frog hearts, whereas saline made up using distilled water (which lacked calcium) could not. Subsequent work revealed that numerous cell biological processes are controlled by calcium (Carafoli 2003). Particularly important was the discovery in the 1950s that calcium triggers skeletal muscle contraction by binding to troponin C and that calcium can be sequestered in the sarcoplasmic reticulum. These studies led to the notion that calcium signals inside cells oscillate: the cytoplasmic calcium concentration increases, a particular effector is activated, and then the calcium signal is reversed to reset the system (see Figs. 1 and ​and2).2). Cells use a “toolkit” of channels, pumps, and cytosolic buffers to control calcium levels (Berridge et al. 2000). Numerous proteins are modulated directly or indirectly by calcium. These include kinases and phosphatases, transcription factors such as NF-AT, and the ubiquitous calcium-binding protein calmodulin (CaM).


\subsection{Sygnałosom wapniowy}



	\subsubsection{Różnorodnośc odpowiedzi na sygnał wapniowy}
	\subsubsection{Białka transportujące wapń}

\subsection{Białka wiążące wapń}

	\subsubsection{Pompy}
	\subsubsection{Kanały}
	\subsubsection{Sensory}
	\subsubsection{Bufory}


\section{Mikrodomeny}

	\subsection{Mikrodomeny mitochondrialno-retikularne}
	\subsection{Mikrodomeny retikularno-plazmatyczne}

\section{Modele homeostazy wapniowej}

\subsection{Modele calo-komórkowe}

\subsection{Modele kompartmentowe}

\subsection{Modele białek transportujących}

\subsection{Modele stochastyczne}
			%arara bibtex
\chapter[Modelowanie sieci sygnałowych]{Modelowanie sieci sygnałowych\\{\large (T.~Lipniacki)}}



% ===================================================================
%       BACK
% ===================================================================
	\backmatter
	%\include{rozdzialy/wnioski}
	\bibliographystyle{ddabbrv}%ThesisStyle
	\bibliography{library1}	

\end{document}

\addcontentsline{toc}{section}{Streszczenie}
\begin{abstract}

Sygnalizacja wapniowa stanowi jeden z najważniejszych szlaków przekazywania sygnału w komórce. Poprzez kontrolę stężenia jonów wapnia regulowane są takie procesy jak moment zapłodnienia, wiele procesów związanych z~różnicowaniem i~morfogenezą, skurczem włókienek mięśniowych, sekrecją hormonów, a nawet niektóre szlaki prowadzące do programowanej śmierci, czyli apoptozy. Wapń w cytozolu stanowi przekaźnik informacji i~może być używany jako sygnał integrujący procesy zachodzące w różnych przedziałach komórki lub jako sygnał zdolny przekazywać sygnał pomiędzy środowiskiem zewnętrznym, a komórką.

Molekularne mechanizmy komórki zaangażowane w kontrolę wapnia obejmują dużą pulę białek wiążących wapń, których cechą charakterystyczną jest występowanie dwóch domen ułatwiających wiązanie tego jonu. Są to odpowiednio domeny C2 oraz tzw. ,,EF-hand'' W zależności od funkcji, białka związane ze szlakiem wapniowym możemy podzielić na transportujące (do tej grupy należą głównie kanały jonowe i wymienniki jonowe), wiążące wapń (bufory i sensory wapniowe) oraz efektorowe (np. kalmodulina). Różnorodność sygnalizacji wapniowej osiągana jest za pomocą dużej ilości białek efektorowych wrażliwych na wapń. Wszystkie białka zaangażowane w przekazywanie sygnału wapniowego tworzą tzw. ,,sygnałosom wapniowy''. Poszczególne rodzaje komórek zawierają różne elementy sygnałosomu, który dostosowany jest do typu komórki i funkcji, jakie pełni. 

Ale nie tylko białka biorą udział w utrzymaniu stanu homeostazy (dynamicznej równowagi) jonów wpania w komórce. Na gospodarkę wapniem mają również wpływ elementy strukturalne wyższego rzędu takie jak mitochondria i retikulum śródplazmatyczne (ER). Dynamiczny charakter tych struktur i ich wzajemne relacje mogą mieć znaczny wpływ na sposób przekazywana sygnału wapniowego w komórce. Mowa tutaj o kompleksach błonowych, które powstają w miejscach kontaktu mitochondrium i ER. Obszary te odkryto już w latach 70-tych, jednak dopiero niedawno zostały intensywnie zbadane za pomocą nowoczesnych technik mikrospokopwych (EM, FRET) i~genetycznych. Odległość między błonami odgraniczającymi w powyższym kompleksie waha się od 10 - 25 nm tworząc fizyczne połączenia przypominające synapsy. Umożliwia to szybsze przekazywanie jonów wapnia z ER do mitochondriów i odwrotnie. Szacuje się, że większość sygnalizacji wapniowej w komórce odbywa się właśnie za pośrdnictwem tych struktur.

Przedmiotem pracy jest przedstawienie najważniejszych elementów i mechnanizmów kształtujących gospodarkę jonami wapnia na najniższym, subkkomórkowym poziomie. Zwłaszcza w~kontekście odkryć ostatnich lat, które w znacznym stopniu zmieniły postrzeganie badaczy na ten aspekt fizjologi komórki. Postęp związany jest m.in. z odkryciami dotyczacymi struktury niektórych kanałów wapniowych oraz funkcjonowania i powtstawania mikrodomen pomiędzy organellami komórkowymi. Odkrycia te rzucają światło na tak ważne zagadnienia jak apoptoza, czy etiologia niektórych chorób neurodegeneracyjnych (np. w chorobie Alzheimera).
\end{abstract}
\chapter[Modelowanie wapnia komórkowego]{Modelowanie homeostazy wapniowej\\{\Large B.~Kaźmierczak}}

\section{Homeostaza wapniowa w komórce}

Jony wapniowe kontrolują wiele różnorodnych procesów komórkowych, takich jak skurcz mięśni, egzocytozę, transkrypcję a nawet apoptozę. Aby uzyskać tak różnorodny wachlarz możliwości komórki wykorzystują zestawy ,,narzędzi'' białkowych, które składają się na tzw. sygnałosom wapniowy \cite{Berridge2012}. Każdy typ komórek charakteryzuje się specyficznym układem białek sensorycznych i efektorowych, które przekazują informacje w dół kaskady informacyjnej, jaką jest wapniowy szlak sygnałowy.




\subsection{Sygnałosom wapniowy}



	\subsubsection{Różnorodnośc odpowiedzi na sygnał wapniowy}
	\subsubsection{Białka transportujące wapń}

\subsection{Białka wiążące wapń}

	\subsubsection{Pompy}
	\subsubsection{Kanały}
	\subsubsection{Sensory}
	\subsubsection{Bufory}


\section{Mikrodomeny}

	\subsection{Mikrodomeny mitochondrialno-retikularne}
	\subsection{Mikrodomeny retikularno-plazmatyczne}

\section{Modele homeostazy wapniowej}

\subsection{Modele calo-komórkowe}

\subsection{Modele kompartmentowe}

\subsection{Modele białek transportujących}

\subsection{Modele stochastyczne}
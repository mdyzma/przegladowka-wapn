\chapter[Modelowanie rozkładu i~transportu wody i~substancji w~organizmie pacjenta]{Modelowanie rozkładu i~transportu wody i~substancji w~organizmie pacjenta\\{\large (M.~Dębowska,~L.~Pstraś,~J.~Poleszczuk,~M.~Pietribiasi, J.~\mbox{Piętka-Stachowska},~A.~Jung)}}

Modelowanie kompartmentowe. Farmakokinetyka. Pozaustrojowe oczyszczanie krwi. Kinetyczny model mocznika. Usuwanie mało- i średnio-cząsteczkowych substancji w hemodializie i dializie otrzewnowej. Modele pseudo-jednokompartmentowe: kinetyka fosforanów w hemodializie. Usuwanie makrocząsteczek: dializa otrzewnowa, zabiegi sztucznej wątroby. Usuwanie nadmiaru wody w czasie hemodializy i dializy otrzewnowej. Model regionalnego przepływu krwi. 